
\documentclass{article}
\usepackage{amsmath, amssymb, graphicx, subcaption, listings,xcolor}  % For \mathbb command

\definecolor{codegreen}{rgb}{0,0.6,0}
\definecolor{codegray}{rgb}{0.5,0.5,0.5}
\definecolor{codepurple}{rgb}{0.58,0,0.82}
\definecolor{backcolour}{rgb}{0.95,0.95,0.92}

\lstdefinestyle{mystyle}{
    backgroundcolor=\color{backcolour},   
    commentstyle=\color{codegreen},
    keywordstyle=\color{magenta},
    numberstyle=\tiny\color{codegray},
    stringstyle=\color{codepurple},
    basicstyle=\ttfamily\footnotesize,
    breakatwhitespace=false,         
    breaklines=true,                 
    captionpos=b,                    
    keepspaces=true,                 
    numbers=left,                    
    numbersep=5pt,                  
    showspaces=false,                
    showstringspaces=false,
    showtabs=false,                  
    tabsize=2
}

\lstset{style=mystyle}


    \title{Practicas MC}
    \date{\today}
    \author{Juan Luis Torres Ramos}

    \begin{document}
        
        % 
        % Portada
        % \pagenumbering{gobble}
        % \maketitle
        % \newpage
        % \pagenumbering{arabic}

        \begin{titlepage}
            \centering
            {\includegraphics[width=1\textwidth]{./Imagenes/logo_universidad_de_granada.png}\par}
            \vspace{1cm}
            {\scshape\Large Escuela Tecnica Superior de Ingenieria Informatica y Telecomunicaciones \par}
            \vspace{2.5cm}
            {\scshape\Huge Practicas Modelos de Computación \par}
            \vspace{1cm}
            {\itshape\Large  Grupo B3 \par} 
            \vfill
            {\Large Juan Luis Torres Ramos \par}
            \vspace{0.5cm}
            {\large 24 Octubre 2023 \par}
            \end{titlepage}


        % introducción
        \section*{Practica 1}
        Encuentra una gramática libre del contexto para generar cada uno de los siguientes lenguajes:

        \begin{enumerate}
            \item $L = \{a^i b^j \, | \, i, j \in \mathbb{N}, \, i \leq j\}$.
            \item $L = \{a^i b^j a^j b^i \, | \, i, j \in \mathbb{N}\}$.
            \item $L = \{a^i b^i a^j b^j \, | \, i, j \in \mathbb{N}\}$.
            \item $L = \{a_i b_i \,|\, i \in \mathbb{N}\} \cup \{b_i a_i \,|\, i \in \mathbb{N\}}$.
            \item $L = \{uu^{-1} \mid u \in \{a, b\}^*\}$.
            \item $L = \{a^i b^j c^{i+j} \, | \, i, j \in \mathbb{N}\}$.    
        \end{enumerate}

        \begin{flushleft}
            donde $\mathbb{N}$ es el conjunto de los numeros naturales incluyendo el 0
        \end{flushleft}

        \vspace{\baselineskip} % paso linea

        % Pasos que voy a seguir para resolver el ejercicio
        \begin{flushleft}
            
            \subsubsection*{Pasos para resolver el ejercicio:}
                        
            \begin{enumerate}
                \item Determinar los símbolos terminales y no terminales.
                \item Determinar el símbolo inicial.
                \item Analizar el lenguaje para determinar qué se pide.
                \item Determinar las reglas de producción.
                \item Comprobar con JFLAP  
            \end{enumerate}
        \end{flushleft}


        \newpage

        % 
        % APARTADO 1

        \subsection*{A. $L = \{a^i b^j \, | \, i, j \in \mathbb{N}, \, i \leq j\}$.}

        \begin{flushleft}
            \begin{enumerate}
                \item Los símbolos terminales serán $\{a,b\}$ y los simbolos no terminales serán $S$ y $B$.
                \item El símbolo inicial será $S$.
                \item Analizar el lenguaje para determinar qué se pide. En este caso, se pide que la cadena tenga un número de $a$ menor o igual que el número de $b$. Por ejemplo, $aabbb$ y $aabb$ pertenecen al lenguaje, pero $aab$ no.
                \item Determino las reglas de producción:
                \begin{itemize}
                    \item $S \rightarrow \epsilon$ (genero la cadena vacía).
                    \item $S \rightarrow aSb$.
                    \item $S \rightarrow Sb$.
                \end{itemize}

                \item compruebo con JFLAP que la gramática es correcta.
                
                \vspace{\baselineskip} % paso linea

                % Imagenes en matriz
                \begin{figure}[h] 
                    \centering
                    \begin{subfigure}[b]{0.45\textwidth}
                        \centering
                        \includegraphics[width=\textwidth]{./Imagenes/produccion1.png}
                        \caption{la producción}
                        \label{fig:label1}
                    \end{subfigure}
                    \hfill
                    \begin{subfigure}[b]{0.45\textwidth}
                        \centering
                        \includegraphics[width=\textwidth]{./Imagenes/grafoaaabb.png}
                        \caption{la cadena $aaabb$}
                        \label{fig:label2}
                    \end{subfigure}
                    \vspace{0.5cm} 
                    \\
                    \begin{subfigure}[b]{0.45\textwidth}
                        \centering
                        \includegraphics[width=\textwidth]{./Imagenes/grafoaabbb.png}
                        \caption{la cadena $aabbb$}
                        \label{fig:label3}
                    \end{subfigure}
                    \hfill
                    \begin{subfigure}[b]{0.45\textwidth}
                        \centering
                        \includegraphics[width=\textwidth]{./Imagenes/grafoaabb.png}
                        \caption{la cadena $aabb$}
                        \label{fig:label4}
                    \end{subfigure}
                    \label{fig:matrix1}
                \end{figure}

                

                 

            \end{enumerate}
        \end{flushleft}

        % 
        % Apartado 2

        \newpage % paso pagina
        \subsection*{B. $L = \{a^i b^j a^j b^i \, | \, i, j \in \mathbb{N}\}$.}
        \begin{flushleft}
            \begin{enumerate}

                \item Los símbolos terminales serán $\{a,b\}$ y los simbolos no terminales serán $S$ y $B$.

                \item El símbolo inicial será $S$.
            
                \item El lenguaje nos pide generar una cadena de 4 caracteres donde primero se generen $a^i b^j$ y luego $a^j b^i$, es decir en los extremos un numero caracteres $i$ y en los caracteres del centro un numero de caracteres $j$. 
                Por ejemplo, $aababb$ y $ab$ pertenecen al lenguaje, pero $aabbab$ no.

                \item Determino las reglas de producción:
                \begin{itemize}
                    \item $S \rightarrow aSb$ (genero mismo numero de caracteres en los extremos).
                    \item $S \rightarrow B$.
                    \item $B \rightarrow bBa$ (genero mismo numero de caracteres en el centro).
                    \item $B \rightarrow \epsilon$ (genero la cadena vacía).
                \end{itemize}

                \item compruebo con JFLAP que la gramática es correcta.

                % Imagenes en matriz
                \begin{figure}[h] 
                    \centering
                    \begin{subfigure}[b]{0.45\textwidth}
                        \centering
                        \includegraphics[width=\textwidth]{./Imagenes/produccion2.png}
                        \caption{la producción}
                        \label{fig:label5}
                    \end{subfigure}
                    \hfill
                    \begin{subfigure}[b]{0.45\textwidth}
                        \centering
                        \includegraphics[width=\textwidth]{./Imagenes/grafo3.png}
                        \caption{la cadena $aabbab$}
                        \label{fig:label6}
                    \end{subfigure}
                    \vspace{0.5cm} 
                    \\
                    \begin{subfigure}[b]{0.45\textwidth}
                        \centering
                        \includegraphics[width=\textwidth]{./Imagenes/grafo1.png}
                        \caption{la cadena $aababb$}
                        \label{fig:label7}
                    \end{subfigure}
                    \hfill
                    \begin{subfigure}[b]{0.45\textwidth}
                        \centering
                        \includegraphics[width=\textwidth]{./Imagenes/grafo2.png}
                        \caption{la cadena $ab$}
                        \label{fig:label8}
                    \end{subfigure}
                    \label{fig:matrix2}
                \end{figure}

            \end{enumerate}
        \end{flushleft}
        
        
        % 
        % Apartado 3
        \newpage % paso pagina
        \subsection*{C. $L = \{a^i b^i a^j b^j \, | \, i, j \in \mathbb{N}\}$.}
        \begin{flushleft}
            \begin{enumerate}
                \item Los símbolos terminales serán $\{a,b\}$ y los simbolos no terminales serán $S$ y $B$.
                \item El símbolo inicial será $S$.
                \item El lenguaje nos pide generar cadenas de 4 caracteres de la forma $abab$ donde los dos primeros caracteres tengan 
                el mismo nuemoor de caracteres y para los dos ultimos caracteres tambien tengan la misma cantidad.Ejemplos de cadenas serían $aabbaabb$ ,$aabbab$ pero no acepta $aaba$
                \item Determino las reglas de producción:
                \begin{itemize}
                    \item $S \rightarrow AA$ (simbolo inicial).
                    \item $A \rightarrow aSb$. (genero $\{a^i b^i | i \in \mathbb{N}\}$).
                    \item $A \rightarrow \epsilon$ (genero la cadena vacía).
                \end{itemize}

                \item compruebo con JFLAP que la gramática es correcta.
                

                % Imagenes en matriz
                \begin{figure}[h] 
                    \centering
                    \begin{subfigure}[b]{0.45\textwidth}
                        \centering
                        \includegraphics[width=\textwidth]{./Imagenes/produccion3.png}
                        \caption{la producción}
                        \label{fig:label10}
                    \end{subfigure}
                    \hfill
                    \begin{subfigure}[b]{0.45\textwidth}
                        \centering
                        \includegraphics[width=\textwidth]{./Imagenes/grafo6.png}
                        \caption{la cadena $aaba$}
                        \label{fig:label11}
                    \end{subfigure}
                    \vspace{0.5cm} 
                    \\
                    \begin{subfigure}[b]{0.45\textwidth}
                        \centering
                        \includegraphics[width=\textwidth]{./Imagenes/grado4.png}
                        \caption{la cadena $aabbaabb$}
                        \label{fig:label12}
                    \end{subfigure}
                    \hfill
                    \begin{subfigure}[b]{0.45\textwidth}
                        \centering
                        \includegraphics[width=\textwidth]{./Imagenes/grafo5.png}
                        \caption{la cadena $aabbab$}
                        \label{fig:label13}
                    \end{subfigure}
                    \label{fig:matrix3}
                \end{figure}

            \end{enumerate}
        \end{flushleft}



        % 
        % Apartado 4
        \newpage % paso pagina
        \subsection*{D. $L = \{a_i b_i \,|\, i \in \mathbb{N}\} \cup \{b_i a_i \,|\, i \in \mathbb{N\}}$.}
        \begin{flushleft}
            \begin{enumerate}
                \item Los símbolos terminales serán $\{a,b\}$ y los simbolos no terminales serán $S$ , $A$ $B$.
                \item El símbolo inicial será $S$ .
                \item Combina dos conjuntos de cadenas: el primero contiene cadenas de la forma $\{a_i b_i \,|\, i \in \mathbb{N}\}$, y el segundo contiene cadenas de la forma $\{b_i a_i \,|\, i \in \mathbb{N}\}$.Las cadenas $aabb$ $bbaa$ lo cumplen mientras $abab$ no lo cumple Lo resolvemos por partes
                \item Determino las reglas de producción:
                
                \begin{itemize}
                    \item Podemos generar $\{a_i b_i \,|\, i \in \mathbb{N}\}. $
                        \subitem $A \rightarrow aAb$ , $A \rightarrow \epsilon$. 
                    \item Por otro lado $\{b_i a_i \,|\, i \in \mathbb{N}\}$.
                        \subitem $B \rightarrow bBa$ , $B \rightarrow \epsilon$ . 
                    \item El lenguaje L se puede generar añadiendo .
                        \subitem $S \rightarrow A$ , $S \rightarrow B$ . 
                \end{itemize}

                \item compruebo con JFLAP que la gramática es correcta.
                % Imagenes en matriz
                \begin{figure}[h] 
                    \centering
                    \begin{subfigure}[b]{0.25\textwidth}
                        \centering
                        \includegraphics[width=\textwidth]{./Imagenes/produccion4.png}
                        \caption{la producción}
                        \label{fig:label14}
                    \end{subfigure}
                    \hfill
                    \begin{subfigure}[b]{0.4\textwidth}
                        \centering
                        \includegraphics[width=\textwidth]{./Imagenes/grafoabab.png}
                        \caption{la cadena $abab$}
                        \label{fig:label15}
                    \end{subfigure}
                    \vspace{0.5cm} 
                    \\
                    \begin{subfigure}[b]{0.4\textwidth}
                        \centering
                        \includegraphics[width=\textwidth]{./Imagenes/grafo7.png}
                        \caption{la cadena $aabb$}
                        \label{fig:label16}
                    \end{subfigure}
                    \hfill
                    \begin{subfigure}[b]{0.4\textwidth}
                        \centering
                        \includegraphics[width=\textwidth]{./Imagenes/grafobbbaaa.png}
                        \caption{la cadena $bbbaaa$}
                        \label{fig:label17}
                    \end{subfigure}
                    \label{fig:matrix4}
                \end{figure}



            \end{enumerate}
        \end{flushleft}
        % 
        % Apartado 5
        \newpage % paso pagina
        \subsection*{E. $L = \{uu^{-1} \mid u \in \{a, b\}^*\}$.}
        \begin{flushleft}
            \begin{enumerate}
                \item Los símbolos terminales serán $\{a,b\}$ y los simbolos no terminales serán $S$.
                \item El símbolo inicial será $S$.
                \item Analizar el lenguaje para determinar qué se pide. En este caso,se pide generar 
                cadenas que son palíndromos formados por caracteres 'a' y 'b'. Cadenas que pertenecen al lenguaje son $abba$ y $bbaabb$ pero no $bbabb$. 
                \item Determino las reglas de producción:
                \begin{itemize}
                    \item $S \rightarrow \epsilon$ (genero la cadena vacía).
                    \item $S \rightarrow aSa$.
                    \item $S \rightarrow bSb$.
                \end{itemize}

                \item compruebo con JFLAP que la gramática es correcta.
                 % Imagenes en matriz
                 \begin{figure}[h] 
                    \centering
                    \begin{subfigure}[b]{0.45\textwidth}
                        \centering
                        \includegraphics[width=\textwidth]{./Imagenes/produccion5.png}
                        \caption{la producción}
                        \label{fig:label18}
                    \end{subfigure}
                    \hfill
                    \begin{subfigure}[b]{0.45\textwidth}
                        \centering
                        \includegraphics[width=\textwidth]{./Imagenes/grafo8.png}
                        \caption{la cadena $bbab$}
                        \label{fig:label19}
                    \end{subfigure}
                    \vspace{0.5cm} 
                    \\
                    \begin{subfigure}[b]{0.45\textwidth}
                        \centering
                        \includegraphics[width=\textwidth]{./Imagenes/grafo9.png}
                        \caption{la cadena $bbaabb$}
                        \label{fig:label20}
                    \end{subfigure}
                    \hfill
                    \begin{subfigure}[b]{0.45\textwidth}
                        \centering
                        \includegraphics[width=\textwidth]{./Imagenes/grafo10.png}
                        \caption{la cadena $abba$}
                        \label{fig:label21}
                    \end{subfigure}
                    \label{fig:matrix5}
                \end{figure}

            \end{enumerate}
        \end{flushleft}

        % 
        % Apartado 6
        \newpage % paso pagina
        \subsection*{F. $L = \{a^i b^j c^{i+j} \, | \, i, j \in \mathbb{N}\}$.}
        \begin{flushleft}
            \begin{enumerate}
                \item Los símbolos terminales serán $\{a,b,c\}$ y los simbolos no terminales serán $S$.
                \item El símbolo inicial será $S$.
                \item En este caso,se pide generar cadenas donde 
                la cantidad de 'a's y 'b's es igual y la cantidad total de 'c's es la suma de las cantidades de 'a' y 'b'
                . Cadenas que cumplen la gramatica son $abbccc$ y $aaabcccc$ pero no $bacc$
                \item Determino las reglas de producción:
                \begin{itemize}
                    \item $S \rightarrow aSc$ (genero la cadena vacía).
                    \item $S \rightarrow B$.
                    \item $B \rightarrow bBc$.
                    \item $B \rightarrow \epsilon$.
                \end{itemize}
                \item compruebo con JFLAP que la gramática es correcta.
                
                % Imagenes en matriz
                \begin{figure}[h] 
                    \centering
                    \begin{subfigure}[b]{0.45\textwidth}
                        \centering
                        \includegraphics[width=\textwidth]{./Imagenes/produccion6.png}
                        \caption{la producción}
                        \label{fig:label22}
                    \end{subfigure}
                    \hfill
                    \begin{subfigure}[b]{0.45\textwidth}
                        \centering
                        \includegraphics[width=\textwidth]{./Imagenes/grado11.png}
                        \caption{la cadena $bacc$}
                        \label{fig:label23}
                    \end{subfigure}
                    \vspace{0.5cm} 
                    \\
                    \begin{subfigure}[b]{0.45\textwidth}
                        \centering
                        \includegraphics[width=\textwidth]{./Imagenes/grafo12.png}
                        \caption{la cadena $abbccc$}
                        \label{fig:label24}
                    \end{subfigure}
                    \hfill
                    \begin{subfigure}[b]{0.45\textwidth}
                        \centering
                        \includegraphics[width=\textwidth]{./Imagenes/grafo13.png}
                        \caption{la cadena $aaabcccc$}
                        \label{fig:label25}
                    \end{subfigure}
                    \label{fig:matrix6}
                \end{figure}


            \end{enumerate}
        \end{flushleft}


        \newpage


        %
        % PRACTICA 2
        \section*{Practica 2}
        Analizadores léxicos, problemas de mineria, trabajo Lex, 2 problema
        \subsection*{Tareas a realizar}
        \begin{enumerate}
        \item Formar un grupo de trabajo compuesto por una, dos o tres personas.
        \item Cada grupo de trabajo debe pensar un problema original de procesamiento de textos. Para la resolución de este problema debe ser apropiado el uso de Lex, o sea, se debe resolver mediante el emparejamiento de cadenas con expresiones regulares y la asociación de acciones a cada emparejamiento.
        \item Cada grupo debe resolver el problema propuesto usando Lex. Se deberá realizar una memoria donde se presente una descripción del problema y su solución, además de entregar electrónicamente los ficheros de texto con la implementación de la solución.
        \item Esta práctica deberá ser entregada antes del día 31 de Diciembre de 2020. Se entregará a través de la plataforma PRADO en un fichero .zip conteniendo todos los archivos de esta práctica. Sólo es necesario que lo entregue uno de los componentes del grupo.
        \end{enumerate}

        \vspace{\baselineskip} % paso linea


        % Pasos que voy a seguir para resolver el ejercicio
        \begin{flushleft}
            
            \subsubsection*{Pasos para resolver el ejercicio:}
                        
            \begin{enumerate}
                \item Descripcion del problema 
                \item solucion 
                \item codigo lex
            \end{enumerate}
        \end{flushleft}

        \newpage

        \subsection*{1. Descripcion del Problema}
        
        Soy un nuevo profesor de la asignatura de Fundamentos de Programacion. Tras corregir varios ejercicios de los alumnos me he dado cuenta que la cantidad de comentarios explicando 
        el codigo va relacionada con la nota del ejercicio. Por lo que he decidido crear un programa que calcule la densidad de comentarios en un codigo fuente en C para evaluar positivamente a los alumnos que comenten su codigo.
    
        \subsubsection*{Densidad de comentarios codigo}
        Tu tarea es desarrollar un programa en Lex que calcule la densidad de comentarios en un código fuente en C. 
        La densidad de comentarios se define como el porcentaje del código total que está ocupado por comentarios. 

        \subsubsection*{Ejemplo}
        El alumno ha entregado su ejercicio de C correspondiente de la asignatura, voy a calcular la densidad de comentarios con la siguiente formula: 
        \[ \text{{Densidad de comentarios}} = \frac{{\text{{Total de letras en un comentario}}}}{{\text{{Total de letras de un comentario}}}} \]
        para resolver el problema tendre que reconocer los comentarios. En C se usa "/* */" y "//", luego calcular el total de letras de un comentario y el total de letras en el codigo, por ultimo usaremos la funcion anterior
        
        \begin{enumerate}
            \item Creo 2 variables globales, para contar letras en el código y en comentarios.
            \item Defino 2 estados \texttt{INCOMMENTBLOCK} y \texttt{INCOMMENTLINE} para manejar por separado los dos tipos de comentarios en C: comentarios en línea y comentarios en bloque.
            \item Defino una función \texttt{contarLetras} que cuenta únicamente letras; no cuenta espacios en blanco, tabuladores, saltos de línea ni retornos de carro.
            \item Defino reglas de Flex para reconocer los comentarios:
                \begin{itemize}
                    \item Si encuentra ``/*'', comienza el subestado \texttt{INCOMMENTBLOCK} y termina con ``*/''.
                    \item Si encuentra ``//'', comienza el subestado \texttt{INCOMMENTLINE} y termina con un salto de línea.
                    \item Dentro del estado \texttt{INCOMMENTBLOCK}, para cualquier carácter que no sea un asterisco (para evitar contar el fin del comentario \texttt{*/}) ni un salto de línea (\texttt{\textbackslash n}). Para comentarios en línea, solo no cuento el salto de línea.
                    \item El texto seleccionado corresponde a la variable \texttt{yytext}, la cual introduzco en mi función \texttt{contarLetras}.
                    \item Imprimo cada comentario encontrado, indicando su tipo y su longitud.
                    \item Para referirme a todo el código, uso \texttt{.|\textbackslash n}, haciendo referencia a cualquier carácter y un salto de línea.

                \end{itemize}
            \item Por último, calculo la densidad de comentarios con la fórmula anterior.
        \end{enumerate}

        \newpage

    
        Entrada
        \lstset{language=C, breaklines=true, basicstyle=\footnotesize}
        \begin{lstlisting}[frame=single]
    
    #include <stdio.h>
    #include <stdlib.h>
            
    // funcion de ejemplo
    void imprimirMensaje() {
        printf("! Hola, mundo!\n");
    }

    /*
        comentarios
        multilinea
    */

    int main{
        // Llamada a la funcion
        imprimirMensaje();
        return 0;
    }
        \end{lstlisting}

        




    \newpage

    
    %
    % PRACTICA 3
    \section*{Practica 3}
    máquinas de estados para codificar y decodificar código Enigma.
    Usando las máquinas de estado finito, en concreto la de Mealy,
    hay que implementar un código para con codificador y después
    se decodifica.
    El codificador y el decodificador, se simula en JFlag como una
    máquina de Mealy



    \end{document}
