

\documentclass{article}
\usepackage{amsmath, amssymb, graphicx, subcaption, tabularx}  % For \mathbb command


    \title{Practicas MC}
    \date{\today}
    \author{Juan Luis Torres Ramos}

    \begin{document}
        
        % 
        % Portada
        % \pagenumbering{gobble}
        % \maketitle
        % \newpage
        % \pagenumbering{arabic}

        \begin{titlepage}
            \centering
            {\includegraphics[width=1\textwidth]{./Imagenes/logo_universidad_de_granada.png}\par}
            \vspace{1cm}
            {\scshape\Large Escuela Tecnica Superior de Ingenieria Informatica y Telecomunicaciones \par}
            \vspace{2.5cm}
            {\scshape\Huge Practicas Modelos de Computación \par}
            \vspace{1cm}
            {\itshape\Large  Grupo B3 \par} 
            \vfill
            {\Large Juan Luis Torres Ramos \par}
            \vspace{0.5cm}
            {\large 24 Octubre 2023 \par}
        \end{titlepage}


        % 
        % Ejercicio 1
        %

        \section*{Ejercicio 1}
            Obtener un modelo de calculo para $L = \{ U \mid U \in \{a, b\}^{\pm} \text{ y } N_a(U) = N_b(U) \}$.
        
            \vspace{\baselineskip} % paso linea

            \textit{
            diapositiva 1 resuelto, pagina 55 del tema 1 de MC\\
            la gramatica generada por este lenguaje se simula en JFLag dando su arbol de derivacion y sacando por pantalla varias salidas
            }
        \newpage




        % 
        % Ejercicio 2
        %

        \section*{Ejercicio 2}
        \begin{enumerate}
            \item determina si la gramática $G = (\{S,A,B\}, \{a,b,c,d\}. P,S)$ donde $P$ es el conjunto regla producción
            genera un lenguaje tipo 3.
            
            \begin{tabularx}{\textwidth}{XXX}
                $S \rightarrow AB$ & $A \rightarrow Ab$ & $A \rightarrow a$\\
                $B \rightarrow cB$ & $B \rightarrow d$\\
            \end{tabularx}
            

            \item obtener el ATDM, (automanta deterministico minimal)
            Obtener el modelo de cálculo más optimo para resolver el problema del apartado A
        \end{enumerate}
        
        \vspace{\baselineskip} % paso linea

        \textit{
        Pregunta resuelta en la diapositiva 91 del tema 1 de MC.
        La gramática generada por este lenguaje se simula en JFlag,
        dando su árbol de derivación y sacando por pantalla varias
        salidas
        }

        \newpage

        % 
        % Ejercicio 3
        %
        \section*{Ejercicio 3}
        construir una expresión regular para las palabras en las que numeros de ceros es par
        
        \vspace{\baselineskip} % paso linea

        Resultado: $$ 1*(01*01*)*$$
        

        \vspace{\baselineskip} % paso linea
        \textit{
            Pregunta resuelta a partir de la diapositiva 104 del tema 2:
            Autómatas Finitos y Expresiones Regulares de MC.
            Las expresiones regulares generadas en cada apartado se
            simulan en JFlag
        }

        \newpage

        % 
        % Ejercicio 4
        %

        \section*{Ejercicio 4}
        \begin{enumerate}
            \item ¿Es $L = \{ U \in \{0, 1\}^* /\  u=u^{-1} \}$. regular? 
            \item Encontrar un modelo de calculo para $L$ 
        \end{enumerate}

        \textit{
            Suponemos que el lenguaje L es regular.
            Entonces satisface el Lema de Bombeo.
            Existe una gramática libre de contexto ya que L(G)=L
            Si es libre de contexto, se crea un autómata con pilas
            crear el automata con pilas y simularlo en JFLAP
        }

    \end{document}
