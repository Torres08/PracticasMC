

\documentclass{article}
\usepackage{amsmath, amssymb, graphicx, subcaption, tabularx}  % For \mathbb command


    \title{Practicas MC}
    \date{\today}
    \author{Juan Luis Torres Ramos}

    \begin{document}
        
        % 
        % Portada
        % \pagenumbering{gobble}
        % \maketitle
        % \newpage
        % \pagenumbering{arabic}

        \begin{titlepage}
            \centering
            {\includegraphics[width=1\textwidth]{./Imagenes/logo_universidad_de_granada.png}\par}
            \vspace{1cm}
            {\scshape\Large Escuela Técnica Superior de Ingeniería Informatica y Telecomunicaciones \par}
            \vspace{2.5cm}
            {\scshape\Huge Examen Modelos de Computación \par}
            \vspace{1cm}
            {\itshape\Large  Grupo B3 \par} 
            \vfill
            {\Large Juan Luis Torres Ramos \par}
            \vspace{0.5cm}
            {\large 15 Enero 2024 \par}
        \end{titlepage}


        % 
        % Ejercicio 1
        %

        \section*{Ejercicio 1}
            \textbf{1. Obtener un modelo de calculo para $L = \{ U \mid U \in \{a, b\}^{\pm} \text{ y } N_a(U) = N_b(U) \}$.}
            
            
            \vspace{\baselineskip} % paso linea

            Para programar la gramática que genera este lenguaje, primero observamos que busca cadenas con el mismo número de $b$'s que de $a$'s $N_a(U) = N_b(U)$. 
            Al aplicar el lema del bombeo, nos dice que el lenguaje no es regular por lo que intentaremos encontrar una gramática libre de contexto. 
            \vspace{\baselineskip} % paso linea

            Tenemos que estudiar todos los casos posibles. Comenzamos con la producción inicial $S$ que genera $a$ y algo más, es decir, $S \rightarrow aX$, donde $X$ es una variable que determinaremos. 
            Esta producción debe garantizar que no se viole la condición $N_a(b) = N_b(b)$. Esto nos lleva a la producción $S \rightarrow aB$, 
            con $B \rightarrow b$, esta ultima produccion simepre va a haber una $b$ de más, asi cumplimos la condicion anterior.
            \vspace{\baselineskip} % paso linea

            De igual manera, razonamos en la dirección opuesta 
            y obtenemos $S \rightarrow bA$ con $A \rightarrow a$. 
            Hasta este punto, hemos definido cuatro producciones.
            \vspace{\baselineskip} % paso linea

            Ahora, estudiamos el caso de la producción $S \rightarrow aB \rightarrow abX$. Este caso nos centraremos cuando $B \rightarrow bX$,como queremos mantener la 
            condicion de B (tener una $b$ de más). Observamos que $B \rightarrow bS$ cumple con la condicion ($S$ genera una a y una b para este caso). 
            De manera análoga, obtenemos $A \rightarrow aS$ para el caso contrario.
            \vspace{\baselineskip} % paso linea

            Estudiamos el caso $B \rightarrow aX$. +
            Necesitamos una $b$ adicional para cumplir la condición de $B$ (tener una $b$ de más), 
            así que definimos $B \rightarrow aBB$, asegurando que la primera 
            $B$ compense una $a$ y luego llamando nuevamente para tener un $B$ 
            adicional. De manera inversa, obtenemos $A \rightarrow bAA$.
            \vspace{\baselineskip} % paso linea

            La gramática libre de contexto resultante es:
            \vspace{\baselineskip} % paso linea
            $G = (V, T, P, S)$ donde $V = \{S, A, B\}$, $T = \{a, b\}$, y $P$ es el conjunto de producciones:
            \vspace{\baselineskip} % paso linea

            \begin{tabularx}{\textwidth}{XXXX}
                $S \rightarrow aB$ & $S \rightarrow bA$ & $B \rightarrow bS$ & $B \rightarrow aBB$\\
                $B \rightarrow b$ & $A \rightarrow a$ & $A \rightarrow aS$ & $A \rightarrow bAA$\\
            \end{tabularx}

           
            \newpage
            \textbf{Comprobacion con JFLAP: }
            generamos la gramatica en JFLAP y comprobamos que genera el lenguaje $L = \{ U \mid U \in \{a, b\}^{\pm} \text{ y } N_a(U) = N_b(U) \}$.
            \begin{figure}[h] 
            \centering
                \centering
                \begin{subfigure}[b]{0.35\textwidth}
                    \centering
                    \includegraphics[width=\textwidth]{./Imagenes/image1.png}
                    \caption{La producción}
                    \label{fig:label1}
                \end{subfigure}
                \hfill
                \begin{subfigure}[b]{0.55\textwidth}
                    \centering
                    \includegraphics[width=\textwidth]{./Imagenes/image2.png}
                    \caption{Comprobación gramatica libre contexto}
                    \label{fig:label2}
                \end{subfigure}
                \vspace{0.5cm} 
                \\
                \begin{subfigure}[b]{0.35\textwidth}
                    \centering
                    \includegraphics[width=\textwidth]{./Imagenes/image3.png}
                    \caption{cadena $aaaabbabbb$}
                    \label{fig:label3}
                \end{subfigure}
                \label{fig:matrix1}
            \end{figure}

            \begin{figure}[!h]
                \centering
                \includegraphics[width=\textwidth]{./Imagenes/image4.png}
                \caption*{(d) árbol de produccion de la cadena $aaaabbabbb$}
                \label{fig:label4}
            \end{figure}
                
        \newpage

        % 
        % Ejercicio 2
        %

        \section*{Ejercicio 2}
        \begin{enumerate}
            \item \textbf{Determina si la gramática $G = (\{S,A,B\}, \{a,b,c,d\}, P,S)$ donde $P$ es el conjunto regla producción
            genera un lenguaje tipo 3:}
            
            \begin{tabularx}{\textwidth}{XXX}
                $S \rightarrow AB$ & $A \rightarrow Ab$ & $A \rightarrow a$\\
                $B \rightarrow cB$ & $B \rightarrow d$\\
            \end{tabularx}
            
            \item \textbf{obtener el ATDM, (automanta deterministico minimal)
            Obtener el modelo de cálculo más optimo para resolver el problema del apartado A}
        \end{enumerate}
        

        \textbf{Apartado 1:}
        Es un tipico problema de optimizacion. Primero nos fijamos en que no es una gramatica regular por producciones
        como $B \rightarrow cB$ o $A \rightarrow Ab$
        Generar un lenguaje de tipo 3 significa que sea libre de contexto.
        Estudio una produccion y genero una cadena de ejemplo:
        $$ S \rightarrow AB \rightarrow AbB \rightarrow AbbB \rightarrow abbcB \rightarrow abbccB \rightarrow abbcccB \rightarrow abbcccd$$ 

        Si nos fijamos genera el lenguaje $L = \{ab^ic^jd : i, j \in \mathbb{N}  \}$ (es decir genera al principio una a, al final una d y entre medias un numero $i$ de b y un numero $j$ de c, en el orden descrito). Vemos si existe una gramatica libre de contexto que genere este lenguaje,
        es decir, optimizamos.

        \vspace{\baselineskip} % paso linea
        Este lenguaje se genera con la gramatica $G = (\{S,A,B\}, \{a,b,c,d\}. P,S)$ donde $P$ es el conjunto regla producción
        \vspace{\baselineskip} % paso linea

        \begin{tabularx}{\textwidth}{XXX}
            $S \rightarrow aB$ & $B \rightarrow C$ & $C \rightarrow d$\\
            $B \rightarrow bB$ & $C \rightarrow cC$\\
        \end{tabularx}

        \vspace{\baselineskip} % paso linea
        \textbf{Comprobacion con JFLAP}
            \begin{figure}[!h] 
            \centering
                \centering
                \begin{subfigure}[b]{0.35\textwidth}
                    \centering
                    \includegraphics[width=\textwidth]{./Imagenes/image8.png}
                    \caption{la producción}
                    \label{fig:label1}
                \end{subfigure}
                \hfill
                \begin{subfigure}[b]{0.55\textwidth}
                    \centering
                    \includegraphics[width=\textwidth]{./Imagenes/image5.png}
                    \caption{gramatica libre contexto}
                    \label{fig:label2}
                \end{subfigure}
                \vspace{0.5cm} 
                \\
                \begin{subfigure}[b]{0.25\textwidth}
                    \centering
                    \includegraphics[width=\textwidth]{./Imagenes/image7.png}
                    \caption{cadena $abbbccd$}
                    \label{fig:label3}
                \end{subfigure}
                \label{fig:matrix1}
            \end{figure}


        \newpage


        \begin{figure}[!h]
            \centering
            \includegraphics[width=\textwidth]{./Imagenes/image6.png}
            \caption*{ (d) árbol de produccion de la cadena  $aabb$}
            \label{fig:label4}
        \end{figure}
        \vspace{\baselineskip} % paso linea


        \textbf{Apartado 2:}
        Quiero crear un autómata no determinístico con transiciones nulas.
        Con la produccion anterior en JFLAP he insertado la gramatica y luego la he convertido a automata finito (Convert-right linear gramar to FA)
        \vspace{\baselineskip} % paso linea

            \begin{figure}[!h] 
            \centering
                \centering
                \begin{subfigure}[b]{0.55\textwidth}
                    \centering
                    \includegraphics[width=\textwidth]{./Imagenes/image10.png}
                    \caption{Producción}
                    \label{fig:label1}
                \end{subfigure}
                \hfill
                \begin{subfigure}[b]{1.05\textwidth}
                    \centering
                    \includegraphics[width=\textwidth]{./Imagenes/image9.png}
                    \caption*{ (b) Automata Finito con transiciones nulas generado}
                    \label{fig:label2}
                \end{subfigure}
                \vspace{0.5cm} 
            \end{figure}
        
        Con un FA, lo he convertido a automata finito deterministico (convert to DFA)
        \begin{figure}[!h]
            \centering
            \includegraphics[width=1\textwidth]{./Imagenes/image11.png}
            \caption*{(c) Automata finito deterministico}
            \label{fig:label4}
        \end{figure}

        Automata finito deterministico minimal
        desde el DFA elo minimozo, vemos que no se puede minimizar por que da el mismo que el anterior
        \begin{figure}[!h]
            \centering
            \includegraphics[width=1.2\textwidth]{./Imagenes/image12.png}
            \caption*{(d) Automata finito deterministico minimal}
            \label{fig:label4}
        \end{figure}

        \newpage 
        \begin{figure}[!h]
            \centering
            \includegraphics[width=1.2\textwidth]{./Imagenes/image13.png}
            \caption*{(e) Ejemplo con la cadena $abbbccd$}
            \label{fig:label4}
        \end{figure}



            

        \newpage

        % 
        % Ejercicio 3
        %
        \section*{Ejercicio 3}
        \textbf{Apartado 1: construir una expresión regular para las palabras en las que numeros de ceros es par}
        
        \vspace{\baselineskip} % paso linea

        $$ 1*(01*01*)*$$
        Forzamos a de esta manera que comience por 1 y que vayan apareciendo ceros de manera par, de dos en dos
        . Simulamos en JFLAP, Convertimos la expresion regular a un automata finito deterministico minimal
        
        \begin{figure}[!h]
            \centering
            \includegraphics[width=0.4\textwidth]{./Imagenes/image19.png}
            \label{fig:label4}
            \caption*{(a) Expresion regular}
        \end{figure}

        \begin{figure}[!h]
            \centering
            \includegraphics[width=0.8\textwidth]{./Imagenes/image14.png}
            \label{fig:label4}
            \caption*{(b) Automata finito no deterministico}
        \end{figure}

        \newpage

        \begin{figure}[!h]
            \centering
            \includegraphics[width=1.2\textwidth]{./Imagenes/image15.png}
            \label{fig:label4}
            \caption*{(c) Automata finito  deterministico}
        \end{figure}

        \begin{figure}[!h]
            \centering
            \includegraphics[width=1.2\textwidth]{./Imagenes/image16.png}
            \label{fig:label4}
            \caption*{(d) Automata finito deterministico minimal}
        \end{figure}

        \newpage
        \begin{figure}[!h]
            \centering
            \includegraphics[width=0.8\textwidth]{./Imagenes/image17.png}
            \label{fig:label4}
            \caption*{(e) Ejemplo de ejecucion}
        \end{figure}

        \vspace{\baselineskip} % paso linea
        \textbf{Apartado 2: Construir una expresion regular para las palabras que contengan a 01100 como subcadena }
         $$(0+1)*0110(0+1)*$$
        Una cadena que comience y termine por cualquier combinacion de 0 y 1 y contenga 0110
        . Simulamos en JFLAP, Convertimos la expresion regular a un automata finito deterministico minimal

        \begin{figure}[!h]
            \centering
            \includegraphics[width=0.4\textwidth]{./Imagenes/image25.png}
            \label{fig:label4}
            \caption*{(a) Expresion regular}
        \end{figure}

        \begin{figure}[!h]
            \centering
            \includegraphics[width=0.55\textwidth]{./Imagenes/image26.png}
            \label{fig:label4}
            \caption*{(b) Automata finito no deterministico}
        \end{figure}

        \newpage

        \begin{figure}[!h]
            \centering
            \includegraphics[width=1.2\textwidth]{./Imagenes/image27.png}
            \label{fig:label4}
            \caption*{(c) Automata finito  deterministico}
        \end{figure}

        \begin{figure}[!h]
            \centering
            \includegraphics[width=1.2\textwidth]{./Imagenes/image28.png}
            \label{fig:label4}
            \caption*{(d) Automata finito deterministico minimal}
        \end{figure}

        \newpage
        \begin{figure}[!h]
            \centering
            \includegraphics[width=0.8\textwidth]{./Imagenes/image29.png}
            \label{fig:label4}
            \caption*{(e) ejemplo de ejecucion}
        \end{figure}

        \textbf{Apartado 3: Construir una expresion regular para el conjunto de palabras que empiezan por 000 y tales que esta subcadena solo se encuentra al principio de la palabra}
        $$ 000(1+10+100)*$$
        Comienza por 000 y luego no vuelve a aparecer 000. Simulamos en JFLAP, Convertimos la expresion regular a un automata finito deterministico minimal

        \begin{figure}[!h]
            \centering
            \includegraphics[width=0.3\textwidth]{./Imagenes/image20.png}
            \label{fig:label4}
            \caption*{(a) Expresion regular}
        \end{figure}

        \begin{figure}[!h]
            \centering
            \includegraphics[width=0.5\textwidth]{./Imagenes/image21.png}
            \label{fig:label4}
            \caption*{(b) Automata finito no deterministico}
        \end{figure}

        \newpage

        \begin{figure}[!h]
            \centering
            \includegraphics[width=1.2\textwidth]{./Imagenes/image22.png}
            \label{fig:label4}
            \caption*{(c) Automata finito  deterministico}
        \end{figure}

        \begin{figure}[!h]
            \centering
            \includegraphics[width=1.2\textwidth]{./Imagenes/image23.png}
            \label{fig:label4}
            \caption*{(d) Automata finito deterministico minimal}
        \end{figure}

        \newpage
        \begin{figure}[!h]
            \centering
            \includegraphics[width=0.8\textwidth]{./Imagenes/image24.png}
            \label{fig:label4}
            \caption*{(e) ejemplo de ejecucion}
        \end{figure}

        


        \newpage

        % 
        % Ejercicio 4
        %

        \section*{Ejercicio 4}
        \begin{enumerate}
            \item \textbf{¿Es $L = \{ U \in \{0, 1\}^* /\  u=u^{-1} \}$. regular? }
            \item \textbf{Encontrar un modelo de calculo para $L$ }
        \end{enumerate}
        El lenguaje $L = \{ U \in \{0, 1\}^* /\  u=u^{-1} \}$ es el conjunto de cadenas binarias que son palíndromos (secuencias de 0 y 1). Un palindromo es 
        una cadena que se lee igual de izquierda a derecha. Para saber si es regular o no aplico el Lema de Bombeo.
        \vspace{\baselineskip} % Salto de línea

        Definimos el Lema de Bombeo: Sea \( L \) un conjunto regular. Entonces, existe un \( n \in \mathbb{N} \) tal que para todo \( z \in L \) con \( |z| \geq n \), se puede expresar como \( z = u v w \) donde:
        \begin{enumerate}
          \item \( |uv| \leq n \)
          \item \( |v| \geq 1 \)
          \item Para todo \( i \in \mathbb{N} \), \( uv^iw \in L \)
        \end{enumerate}
        Además, el número de estados de cualquier autómata que acepta el lenguaje \( L \) es al menos \( n \).
        \vspace{\baselineskip} % Salto de línea

        La idea básica del lema de bombeo es que para cualquier cadena lo suficientemente larga, 
        puedes "bombear" o repetir una parte de la cadena y 
        aún obtener una cadena en el lenguaje. 
        Este lema siempre es verdadero para lenguajes regulares. 
        Si no es verdadero para un lenguaje, entonces ese lenguaje 
        no es regular. Estudiamos el caso del contraejemplo.
        
        \vspace{\baselineskip} % Salto de línea
        
        
        Contraejemplo: \( L \) es regular y satisface el lema de bombeo.
        \[ \exists n \in \mathbb{N},\quad \forall Z \in L,\quad |Z| \geq n,\quad Z = 0^n 1^n 0^n = uvw \]
        
        
        Aplicamos \( |uv| \leq n \):
        \[ uv = 0^k, \quad v = 0^l, \quad w = 0^{n-k-l}1^n0^n \]
        
        \( |v| \geq 1 \) implica que \( v = 0^l \) y \( l \geq 1 \).
        \vspace{\baselineskip} % Salto de línea

        Para \( i \geq 2 \), verificamos \( uv^iw \in L \):
        \[ i = 2\quad  uv^2w = 0^k 0^{2l} 0^{n-k-l} 1^n 0^n =\quad 0^{n+l} 1^n 0^n \notin L \]
           
        Hemos alcanzado una contradicción con el lema de bombeo, por lo que \( L \) no es regular.
        
        \vspace{\baselineskip} % paso linea

        
        \newpage
        \textbf{Apartado 2: Obtener un modelo de calculo para $L$}         \vspace{\baselineskip} % paso linea

         El modelo es no deterministico, lo que significa que no diferencia cuando cambia entre estados. 
         Simulo todos los posibles casos.
         $q0$ se encarga de la parte de la derecha del palindromo y $q1$ de la parte izquierda 
         Cuando el numero es par el cambio se hace con el 0 o 1. Cuando el numero es impar el cambio se haria con $\lambda$ 
         
        \begin{figure}[!h]
            \centering
            \includegraphics[width=\textwidth]{./Imagenes/image30.png}  
            \label{fig:label4}
            \caption*{(a) caso impar 10001}
        \end{figure}

        \begin{figure}[!h]
            \centering
            \includegraphics[width=\textwidth]{./Imagenes/image31.png}
            \label{fig:label4}
            \caption*{(b) caso par 0110}
        \end{figure}
    \end{document}
